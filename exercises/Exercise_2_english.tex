% LaTeX source for textbook ``How to think like a computer scientist''
% Copyright (C) 1999  Allen B. Downey
% Copyright (C) 2009  Thomas Scheffler

\begin{exercise}
\label{ex.date}

\begin{enumerate}

\item Create a new program named {\tt MyDate.c}.  Copy or
type in something like the "Hello, World" program and make
sure you can compile and run it.

\item Following the example in Section~\ref{output variables}, write a program
that creates variables named {\tt day}, {\tt month}
and {\tt year}
What type is each variable?

Assign values to those variables that represent today's date.

\item Print the value of each variable on a line by itself.  This is
an intermediate step that is useful for checking that everything is
working so far.

\item Modify the program so that it prints the date in standard
American form: {\tt {\tt mm/dd/yyyy}}.

\item Modify the program again so that the total output is:

\begin{verbatim}
American format:
3/18/2009
European format:
18.3.2009
\end{verbatim}

\end{enumerate}

The point of this exercise is to use the output function {\tt printf} to display
values with different types, and to
practice developing programs gradually by adding a few statements
at a time.



\end{exercise}


\begin{exercise}

\begin{enumerate}

\item Create a new program called {\tt MyTime.c}.  From now
on, I won't remind you to start with a small, working program,
but you should.

\item Following the example in Section~\ref{operators}, create variables
named {\tt hour}, {\tt minute} and {\tt second}, and assign
them values that are roughly the current time.  Use a 24-hour
clock, so that at 2pm the value of {\tt hour} is 14.

\item Make the program calculate and print the number of
seconds since midnight.

\item Make the program calculate and print the number of
seconds remaining in the day.

\item Make the program calculate and print the percentage of
the day that has passed.

\item Change the values of {\tt hour}, {\tt minute} and {\tt second}
to reflect the current time (I assume that some time has elapsed), and
check to make sure that the program works correctly with different
values.

\end{enumerate}

The point of this exercise is to use some of the arithmetic
operations, and to start thinking about compound entities like the
time of day that are represented with multiple values.  Also,
you might run into problems computing percentages with {\tt ints},
which is the motivation for floating point numbers in the next
chapter.

HINT: you may want to use additional variables to hold values
temporarily during the computation.  Variables like this, that
are used in a computation but never printed, are sometimes called
intermediate or temporary variables.

\end{exercise}

