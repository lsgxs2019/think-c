% LaTeX source for textbook ``How to think like a computer scientist''
% Copyright (C) 1999  Allen B. Downey
% Copyright (C) 2009  Thomas Scheffler

%%%%%%%%%%%%%%%%%%%%%%%%%%%%%%%%%%%%%%%%%%

\begin{exercise}

Section~\ref{Structures as parameters} defines the function {\tt PrintPoint()}. 
The argument of this function is passed along as a value (\emph{call-by-value}).

\begin{enumerate}
\item Change the definition of this function, so that it only passes a reference to 
the structure to the function for printing (\emph{call-by-reference}).

\item Test the new function with different values and document the results. 
\end{enumerate}


\end{exercise}


%%%%%%%%%%%%%%%%%%%%%%%%%%%%%%%%%%%%%%%%%%

\begin{exercise}

Write a program, that defines a struct \texttt{Person\_t}.
This struct should contain two members. The first member should be a string
of sufficient length, to contain the name of a person. The second member should
be an integer value containing the persons age. 

\begin{enumerate}
\item Define two struct variables \texttt{person1} and \texttt{person2}.\\ 
Initialize the two struct variables with suitable values. The first person
should be called Betsy. Betsy should be 42 years old.\\
The second person should be named after yourself and should also be as old as yourself.


\item Write a function \texttt{PrintPerson()}. The function should take a struct variable of type
\texttt{Person\_t} as an argument and print the name of the person and the corresponding age.
\item Write a function \texttt{HappyBirthday()}. The function should increase the age of the 
corresponding person by one year and print out a birthday greeting.\\
\textbf{Hint: }You need to passes a reference to the structure to the function (\emph{call-by-reference}). 
If your function uses \emph{call-by-value}, the age only changes in the local copy of the struct and
the changes you have made are lost, once the function terminates. Try it out, if you do not believe me!
\end{enumerate}


\end{exercise}




%%%%%%%%%%%%%%%%%%%%%%%%%%%%%%%%%%%%%%%%%%

\begin{exercise}

Most computer games can capture our interest only when their actions are
non-predictable, otherwise they become boring quickly.
Section~\ref{Random numbers} tells us how to generate random numbers in
in C. 


Write a simple game, where the computer chooses an arbitrary number in
the range between 1 and 20. You will then be asked to guess the
number chosen by the Computer.

To give you a hint the computer should answer your guess in the following
way: in case your guess was lower than the number of the computer, the
output should be: \emph{My number is larger!} \\
If you guess was higher than
the number of the computer, the output should read: \\
\emph{My number is smaller!}

It is necessary to seed the random number generator when you start your program
(cf. Section~\ref{Random seeds}.
You could use the {\tt time()} function for this. It returns the actual time measured
in seconds since 1971.

\begin{verbatim}
    srand(time(NULL));   /*Initialisation of the random number generator*/
\end{verbatim}

When you found the right answer, the computer should congratulate you. The 
program should also display the number of tries that where needed in order
to guess the number. 

The program should also keep the a `High-Score', that gets updated once
our number of trials is lower than any previous try.
The High-Score (the number of minimal guesses) should be stored in a
 {\tt struct}, together with the name of the player.

The \texttt{High-Score()} function should ask for your name and store it, when 
your current number of tries is lower than the previous High-Score value. 

The program then gives you the chance to play again or stop the game by pressing
 'q' on the keyboard.



\end{exercise}


